\documentclass[10pt]{beamer}

\usetheme[progressbar=frametitle]{metropolis}

\definecolor{ubcBlue}{RGB}{12,35,68}
\definecolor{ubcBlue1}{RGB}{0,85,183}
\definecolor{ubcBlue2}{RGB}{0,167,225}
\definecolor{ubcBlue3}{RGB}{64,180,229}
\definecolor{ubcBlue4}{RGB}{110,196,232}
\definecolor{ubcBlue5}{RGB}{151,212,223}

% \setbeamercolor{normal text}{bg=ubcBlue1}
\setbeamercolor{alerted text}{bg=ubcBlue1, fg = ubcBlue}
\setbeamercolor{example text}{fg=ubcBlue1, bg=ubcBlue1}
\setbeamercolor{title separator}{fg = ubcBlue, bg=ubcBlue}
\setbeamercolor{progress bar}{bg=ubcBlue4, fg=ubcBlue1}
\setbeamercolor{progress bar in head/foot}{bg=ubcBlue4, fg=ubcBlue1}
\setbeamercolor{progress bar in section page}{bg=ubcBlue4, fg=ubcBlue1}
\setbeamercolor{frametitle}{bg=ubcBlue}


\usepackage{appendixnumberbeamer}

\usepackage{booktabs,dcolumn,caption}
\usepackage[scale=2]{ccicons}
\usepackage{apacite}
\usepackage{pgfplots}
\usepgfplotslibrary{dateplot}

\usepackage{xspace}
\newcommand{\themename}{\textbf{\textsc{metropolis}}\xspace}

\bibliographystyle{apacite}

\makeatletter
\newsavebox{\mybox}
\setbeamertemplate{frametitle}{%
  \nointerlineskip%
  \savebox{\mybox}{%
      \begin{beamercolorbox}[%
          wd=\paperwidth,%
          sep=0pt,%
          leftskip=\metropolis@frametitle@padding,%
          rightskip=\metropolis@frametitle@padding,%
        ]{frametitle}%
      \metropolis@frametitlestrut@start\insertframetitle\metropolis@frametitlestrut@end%
      \end{beamercolorbox}%
    }
  \begin{beamercolorbox}[%
      wd=\paperwidth,%
      sep=0pt,%
      leftskip=\metropolis@frametitle@padding,%
      rightskip=\metropolis@frametitle@padding,%
    ]{frametitle}%
  \metropolis@frametitlestrut@start\insertframetitle\metropolis@frametitlestrut@end%
  \hfill%
  \raisebox{-\metropolis@frametitle@padding}{\includegraphics[height=\dimexpr\ht\mybox+\metropolis@frametitle@padding\relax]{2_2016_UBCNarrow_Signature_ReverseCMYK}}%
    \hspace*{-\metropolis@frametitle@padding}
  \end{beamercolorbox}%
}
\makeatother

\title{Learning while Buffering: Impact of Bandwidth Quality on Educational Outcomes}
\date{\today}
%\date{}
\author{Mukund Sundararajan}
\institute{ECON 490}
% \titlegraphic{\hfill\includegraphics[height=1.5cm]{logo.pdf}}

\begin{document}

\maketitle

% \begin{frame}{Table of contents}
%   \setbeamertemplate{section in toc}[sections numbered]
%   \tableofcontents[hideallsubsections]
% \end{frame}

\section{Introduction}

\begin{frame}[fragile]{Background}
  \begin{itemize}[<+->]
    \item Experience from the ongoing Covid-19 Pandemic
    \item Past research on usage patterns
    \item Past research on availability and long term outcomes
  \end{itemize}
  \vfill
  \only<.(1)>  {Lots of evidence to show that student access to internet has a positive effect on education}

\end{frame}

\begin{frame}[fragile]{Research Question}
  \begin{itemize}[<+->]
    \item Strong urban/rural divide in bandwidth accessibility - Digital Divide
    \item Video tutorials impractical when faced with low bandwidth or metered connections
  \end{itemize}
  \vfill
  \only<.(1)> {To what extent does having access to high quality bandwidth impact educational outcomes?}
\end{frame}

\section{Data and Methods}

\begin{frame}{Data}
  Main data sources of interest:
  \begin{itemize}[<+->]
    \item Fraser School Reports
    \item National Broadband Data
    \item 2016 Canadian Census
  \end{itemize}
\end{frame}

\begin{frame}{Methods}
  Preliminary OLS with specification:
  \begin{align*}      
    \text{Average\_Exam\_Mark}_i &= \beta_0 + \beta_1(\text{combined\_50\_10}_i) + \beta_2(\text{combined\_25\_5}_i) + \\
                                 & \beta_3(\text{combined\_10\_2}_i) + \beta_4(\text{combined\_5\_1}_i) + \beta_5(\text{public}_i) + \\
                                 & \beta_6(\text{percent\_ESL}_i) + \beta_7(\text{percent\_special\_needs}_i) + \epsilon_i
  \end{align*}
\end{frame}

\section{Results and Analysis}
\newcolumntype{d}[1]{D{.}{.}{#1}} % "decimal" column type
\renewcommand{\ast}{{}^{\textstyle *}} % for raised "asterisks"
\begin{frame}{Results}
  \begin{table}[H]
    \setlength\tabcolsep{0pt} % let LaTeX figure out amount of inter-column whitespace
    \begin{tabular*}{\textwidth}{@{\extracolsep{\fill}} l *{3}{d{2.4}} }
    \toprule
     & \multicolumn{3}{c}{OLS Regression Results} \\
    \midrule
    combined\_5\_1  & 77.4835\ast & 86.4056\ast & 87.1048\ast \\
                  & (6.640) & (6.467) & (6.565)\\
    combined\_10\_2 & 65.7089\ast & 74.4091\ast & 74.6567\ast \\
                  & (9.762) & (9.338) & (9.572)\\
    combined\_25\_5
                  & 86.0053\ast & 93.4669\ast & 93.8178 \\
                  & (4.063) & (4.065) & (4.360)\\
    combined\_50\_10
                  & 70.7386\ast & 74.9946\ast	 & 75.7003\ast \\
                  & (0.428) & (0.876) & (1.023)\\
    public
                  &  & -5.2427\ast & -4.1612\ast \\
                  &  & (0.958) & (1.145)\\
    school\_controls
                  &  No & No & Yes \\

    \bottomrule
    \end{tabular*}
  \end{table}
\end{frame}

\end{document}

